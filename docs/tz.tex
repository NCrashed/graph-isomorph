\documentclass[russian,utf8,emptystyle]{eskdtext}

\newcommand{\No}{\textnumero} % костыль для фикса ошибки

\ESKDdepartment{Федеральное государственное бюджетное образовательное учреждение высшего профессионального образования}
\ESKDcompany{Московский государственный технический университет им. Н. Э. Баумана}
\ESKDclassCode{23 0102}
\ESKDtitle{АИС поиска алгоритмов распознавания изоморфизма графов с помощью генетического программирования}
\ESKDdocName{Техническое задание}
%\ESKDsignature{Вариант 8Б}
\ESKDauthor{Гуща~А.~В.}
\ESKDtitleApprovedBy{~}{~\underline{\hspace{2.5cm}}}
\ESKDtitleAgreedBy{~}{~\underline{\hspace{2.5cm}}}
\ESKDtitleDesignedBy{Студент группы ИУ5-82}{Гуща~А.~В}
 
\usepackage{multirow}
\usepackage{tabularx}
\usepackage{tabularx,ragged2e}
\renewcommand\tabularxcolumn[1]{>{\Centering}p{#1}}
\newcommand\abs[1]{\left|#1\right|}

\begin{document}
\maketitle
\tableofcontents
\newpage

\section{Наименование}
Автоматизированная информационная система поиска алгоритмов распознавания изоморфизма графов с помощью генетического программирования. Далее используется сокращение: программа.

\section{Основание для разработки}
Основанием для разработки является задание на курсовой проект, подписанное руководителем курсового проекта.

\section{Исполнитель}
Студент группы ИУ5-82 Гуща Антон Валерьевич

\section{Назначение и цель разработки}
Назначением разработки является предоставление пользователю инструмента поиска и анализа алгоритмов определения отношения изоморфизма ориентированных графов. 

Целью разработки является нахождение алгоритмов проверки отношения изоморфизма для ориентированных графов, отбор наилучших алгоритмов и анализ полученного решения. Программа должна автоматически проводить поиск и отбор алгоритмов и предоставлять пользователю графическую и текстовую информацию о промежуточных результатах для детального анализа пользователем. Автоматические процедуры, производимые программой, должны быть настраиваемыми пользователем для достижения результатов в кратчайшие сроки с желаемым качеством.

\section{Требования к программе}
\subsection{Задачи, подлежащие решению}
\begin{enumerate}
\item Исследование предметной области проектирования
\item Определение функциональных задач
\item Изучение метода <<Генетическое программирование>>
\item Разработка проблемно-ориентированного языка для внутреннего представления программ
\item Выбор и обоснование критериев качества оценки работы найденных алгоритмов
\item Разработка схемы данных
\item Разработка алгоритмов программы
\item Разработка программы
\item Отладка программы
\item Разработка графического интерфейса пользователя
\item Тестирование программы
\item Разработка конструкторской и эксплуатационной документации
\end{enumerate}

\subsection{Требования к программному изделию}
\begin{enumerate}
\item Просмотр используемого проблемно-ориентированного языка
\item Просмотр и редактирование параметров автоматических процессов эволюции
\item Сохрание и загрузка параметров эволюции и промежуточных результатов
\item Управление процессом эволюции: запуск, постановка на паузу, возобновление после паузы, остановка
\item Просмотр входных данных алгоритмов во время эволюционного процесса
\item Просмотр промежуточных параметров эволюционного процесса
\item Просмотр статуса работы эволюционного процесса
\item Просмотр промежуточных результатов после каждого эволюционного цикла:
\begin{itemize}
\item Просмотр текущих алгоритмов, найденных программой
\item Просмотр значения оценки качества алгоритма
\item Просмотр исходного кода алгоритма
\item Просмотр графического изображения исходного кода алгоритма для проведения ручного анализа
\end{itemize}
\end{enumerate}

\subsection{Требования к архитектуре программного изделия}
Программа должна работать в окружении операционной системы GNU/Linux и поддерживать вывод через графическую систему X Window System.

\subsection{Требования к входным и выходным данным}
\begin{enumerate}
\item Все входные данные вводятся через графический интерфейс пользователя. К ним относятся:
\begin{enumerate}
\item Параметры процесса эволюции
\item Сохраненные параметры и популяции
\end{enumerate}
\item Все выходные данные должны предоставляться пользователю через графический интерфейс пользователя. К ним относятся:
\begin{enumerate}
\item Промежуточные значения процесса эволюции: номер поколения, максимальная и средняя приспособленность.
\item Найденные алгоритмы, представленные в текстовой и графической форме.
\end{enumerate}
\end{enumerate}

\subsection{Требования к надежности}
Программа должна обеспечить поиск алгоритмов проверки отношения изоморфизма ориентированных графов, не должна выдавать ошибок, не предусмотренных работой программы.

\subsection{Требования к составу технических средств}
Программное обеспечение:
\begin{itemize}
\item Операционная система GNU/Linux с версией ядра не ниже 3.0
\item Оконная система X Window System не ниже версии X11R7.3
\item Библиотека элементов интерфейса GTK+ не ниже версии 3.10
\end{itemize}

Аппаратное обеспечение:
\begin{itemize}
\item Процессор, поддерживающий архитектуру x86\underline{~}64 с тактовой частотой не менее 1.5 ГГц
\item Оперативная память от 1 Гб
\item Графический ускоритель и монитор, способные отображать графический интерфейс операционной системы
\item Устройства ввода: мышь и клавиатура
\end{itemize}

\section{Этапы разработки}
\begin{center}
\begin{tabularx}{\textwidth}{c|X|X|X}
№ & Наименование этапа & Форма завершения & Срок \\ 
\hline 
1 & Изучение предметной области & Рабочие материалы & Январь 2014 г. \\ 
\hline 
2 & Разработка технического задания & Согласованное и утвержденное техническое задание & Февраль 2014 г. \\ 
\hline 
3 & Разработка DSL & Рабочие материалы & Март 2014 г. \\ 
\hline 
4 & Разработка программы & Рабочие материалы & Апрель 2014 г. \\ 
\hline 
5 & Тестирование программы &   & Апрель 2014 г. \\ 
\hline 
6 & Разработка документации & Техническая документация (в соответствии с п. 7 ТЗ) & Май 2014 г. \\ 
\hline 
7 & Сдача и приемка программы &   & Май 2014 г. \\
\end{tabularx} 
\end{center}

\section{Техническая документация, предъявляемая по окончании работы}
\begin{enumerate}
\item Техническое задание
\item Расчетно-пояснительная записка
\item Текст программы
\item Программа и методика испытаний
\item Руководство пользователя
\item Графические материалы (6 листов А1):
\begin{enumerate}
\item Общие сведения о спроектированной АИС
\item Структурная схема АИС
\item Описание проблемно-ориентированного языка генетического программирования
\item Основные алгоритмы обработки информации
\item Диаграмма основных классов АИС
\item Схема графа диалога
\end{enumerate}
\end{enumerate}

\section{Порядок приема работы}
Приемка работы осуществляется в соответствии с документом <<Программа и методика испытаний>>.

\section{Дополнительные условия}
Данное Техническое Задание может дополняться и изменяться в установленном порядке.
\end{document}


