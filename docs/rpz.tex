\documentclass[russian,utf8,emptystyle]{eskdtext}

\newcommand{\No}{\textnumero} % костыль для фикса ошибки

\ESKDdepartment{Федеральное государственное бюджетное образовательное учреждение высшего профессионального образования}
\ESKDcompany{Московский государственный технический университет им. Н. Э. Баумана}
\ESKDclassCode{23 0102}
\ESKDtitle{АИС поиска алгоритмов распознавания изоморфизма графов с помощью генетического программирования}
\ESKDdocName{Расчетно-пояснительная записка}
\ESKDauthor{Гуща~А.~В.}
\ESKDtitleApprovedBy{~}{~\underline{\hspace{2.5cm}}}
\ESKDtitleAgreedBy{~}{~\underline{\hspace{2.5cm}}}
\ESKDtitleDesignedBy{Студент группы ИУ5-82}{Гуща~А.~В}

\usepackage{multirow}
\usepackage{tabularx}
\usepackage{tabularx,ragged2e}
\usepackage{pdfpages}
\renewcommand\tabularxcolumn[1]{>{\Centering}p{#1}}
\newcommand\abs[1]{\left|#1\right|}

\begin{document}
\includepdf[pages={1}]{title.pdf}

\newpage
\tableofcontents
\newpage

\section{Введение}
Задача проверки изоморфизма графов является актуальной и исключительно привлекательной проблемой в наше время. Нахождение эффективного алгоритма, который за полиномиальное время позволит отвечать на данный вопрос, положительным образом повлияет на такие прикладные задачи как:
\begin{itemize}
\item Поиск химических соединений по базам данных в хемоинформатике и математической химии
\item Верификация различных представлений электронной схемы в автоматизации проектирования электронных схем
\item Выделение общих подвыражений в оптимизации программ
\item Сопоставление графов знаний, содержащихся в семантических сетях
\end{itemize}
Уникальность данной задачи в том, что это одна из двух задач (и одна из 12, перечисленных в \cite{GareyAndJohnson1979}), для которых класс сложности не был определен. Задача проверки изоморфизма графов принадлежит классу NP задач, но не доказано, что она является NP-полной задачей, и не найден алгоритм, решающий ее за полиномиальное время.

В 60-х~--- 80-х годах неоднократно предпринимались попытки решить данную задачу, но они не увенчались успехом. На данный момент лучший алгоритм имеет временную оценку сложности $2^{O(\sqrt{n log(n)})}$ \cite{Johnson2005} \cite{BabaiCodenotti2008}.

В наши дни информационные технологии все больше используются как научные инструменты (яркий пример - решение проблемы четырех красок \cite{FourColourProblem}). С ростом вычислительной мощности растет актуальность использовать автоматические методы поиска решения, например, метод генетического программирования. В данной работе разработан инструмент, спроектированный производить поиск решения задачи проверки отношения изоморфизма для ориентированных графов с выводом преобразованных в графическую форму промежуточных результатов для анализа человеком.

\newpage
\section{Конструкторская часть}
\subsection{Общетехническое обоснование разработки}
\subsubsection{Постановка задачи проектирования}

\subsubsection{Описание предметной области}

\subsubsection{Перечень процессов, подлежащих автоматизации}

\subsubsection{Выбор и обоснование критериев качества}

\subsubsection{Анализ аналогов и прототипов}

\newpage
\subsection{Разработка программного изделия}
\subsubsection{Разработка структуры программного изделия}

\subsubsection{Особенности выбранных технологий}

\subsubsection{Архитектруа программы. UML-диаграмма классов}

\subsubsection{Выбор программных средств}

\subsubsection{Выбор аппаратных средств}

\subsubsection{Разработка основных алгоритмов обработки информации}

\newpage
\subsection{Технологическая часть}
\subsection{Разработка интерфейса взаимодействия с пользователем}

\subsection{Разработка форматов входных и выходных данных программы}
\subsubsection{Разработка форм входных данных}

\subsubsection{Разработка форм выходных данных}

\newpage
\section{Исследовательская часть}

\newpage
\section{Заключение}

\newpage
%\addcontentsline{toc}{chapter}{\bibname}
\bibliographystyle{utf8gost705u}  %% стилевой файл для оформления по ГОСТу
\bibliography{biblio}     %% имя библиографической базы (bib-файла) 

\end{document}